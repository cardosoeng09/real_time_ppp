\chapter{Apêndice C - Dificuldades}
\noindent

\section{Dificuldades enfrentadas o longo do projeto}

\begin{itemize}
    \item \textit{Porta COM}: Os notebooks mais modernos não possuem entrada COM e a conexão do receptor é por meio desta. Assim, necessitou-se buscar um adaptador COM para USB;
    \item \textit{Atualização de firmware}: A empresa responsável pelo produto veio ao IME realizar a atualização de firmware dos receptores para a correta configuração dos mesmos;
    \item \textit{Configurações da conexão}: Para a correta comunicação do receptor e o notebook, foram necessárias realizar diversas configurações, as quais foram retiradas do manual do GR-5 e testadas. A taxa de transmissão dos dados é uma seleção crucial para ocorrer a comunicação e dentre as opções, apenas uma era a correta;
    \item \textit{Bluetooth}: Ao alterar a taxa de transmissão de dados inúmeras vezes, alteramos a configuração do bluetooth, o qual permite a comunicação entre a Base e o Rover. Para correção, foi preciso investigar o manual e consultas web;
    \item \textit{Mensagem de ''Wrong Epoch''}: Ocorreram divergências entre o horário padrão software BNC e das mensagens advinda dos casters e estações. Ao buscar a solução desse problema, descobriu-se que o BNC possui uma diferença de quatro horas ao fuso padrão do notebook;
    \item \textit{PPP-Wizard}: Pela característica nativa da modificação do BNC pelo código fonte, ocorreram dificuldades no manuseio dessa funcionalidade pela janela de comando, demandando tempo e configurações adicionais.
    
    

\end{itemize}