\chapter{Conclusão}

O desenvolvimento deste projeto gerou especificações técnicas e instruções para a realização dos métodos RTPPP, PPP-RTK e RTK com ferramentas de softwares livres e com receptor GNSS GR-5. Além disso, \textit{scripts} foram implementados para a extração das informações geradas por tais softwares possibilitando uma melhor visualização e análise dos dados. Tais especificações e códigos encontram-se no apêndice deste projeto. A análise aprofundada dos métodos de posicionamento espacial tratados neste projeto, apura a aplicação da teoria no mundo real além de demonstrar sua eficácia em relação à acurácia esperada e atendida. 

O RTPPP realizado em uma estação da RBMC mostrou que a incerteza das coordenadas permanece em valores inferiores a 10 cm e sua convergência para tais valores foi aproximadamente 40 minutos. Além da utilização da estação da RBMC, fez-se um RTPPP com o receptor GNSS conectado via porta serial e tal atividade gerou o conhecimento para realizar a conexão da maneira correta e eficiente; as orientações para realizar essa conexão se encontram nos apêndices deste projeto. Os valores das incertezas das coordenadas e tempo de convergência são similares ao levantamento RTPPP realizado com a estação da RBMC. Ao realizar o RTPPP com estação cinemática evidenciou-se sua operacionalidade e sua realização inteiramente com ferramentas livres. Ao comparar as coordenadas obtidas com as pós-processadas na ferramenta IBGE-PPP tem-se que no levantamento estático a resultante da diferenças das coordenadas apresenta valores próximos a 30 cm na maior parte do tempo, enquanto que no levantamento cinemático a mesma comparação apresenta valores próximos a 2 m (valores altos devidos o pouco tempo de levantamento que não possibilitou a devida convergência).

O PPP-RTK realizado em comparação com o RTPPP mostra que sua convergência é mais rápida, além de possuir valores de incerteza pŕoximos a 3 cm, enquanto que o RTPPP possui 5 cm. Tal fato dá-se em função da utilização da solução fixa de ambiguidades do PPP-RTK e não da solução \textit{float} de ambiguidades (RTPPP). Ainda, a realização do RTK ''clássico'' possibilitou a geração de informações que auxiliam e agilizam a configuração do receptor GR-5 (apêndice).


 As dificuldades encontradas foram amplamente analisadas e solucionadas na medida do possível, como as configurações dos receptores desde o início, a aquisição de materiais para a conexão (cabos) e o entendimento do funcionamento dos métodos. Outras dificuldades exploradas não tiveram solução completa, como a questão do recebimento de dados pelo receptor por meio de internet 4G via ''\textit{SIM card}'' e um novo levantamento RTK. Por questões de gestão de tempo e priorização dos objetivos principais, estas tarefas podem ser exploradas em trabalhos futuros, valendo-se do escopo deste trabalho. A realização deste projeto materializa a possibilidade de utilizar meios modernos (receptores recém lançados) e acessíveis (plataformas e softwares livres) para as diversas empregabilidades citadas
Assim, o anseio de aplicar produtos GNSS em projetos de engenharia é crescente e factível para trabalhos futuros para explorar outros detalhes não abordados no presente relatório.