\chapter{Materiais e Métodos}


\section{Equipamentos e Softwares}
\label{materiais}

A seguir, serão apresentados os equipamentos e recursos utilizados para o processamento em tempo real e o pós-processamento.

\begin{itemize}

\item Processamento: As conexões aos Casters são feitas via software, o qual realiza o processamento dos dados em tempo real. Será utilizada a plataforma de software livre” BKG Ntrip Client (BNC) 2.12.9”. Já para o pós-processamento, ''Topcon Tools 8.2.3.19157'' e ''Magnet Tools''.    

\item Software BKG Ntrip Client (BNC): Recupera, decodifica, converte e processa fluxos de dados GNSS em tempo real (BNC, 013). Além da principal funcionalidade de processamento em tempo real, ainda possui algumas funcionalidades de pós-processamento. Tal ferramenta foi desenvolvida no âmbito do IAG para o sistema EUREF e para o IGS. O BKG Ntrip Client (BNC) foi escrito uma licensa aberta, a GNU General Public License (GPL), desta forma seu código é aberto e está disponivel em http://software.rtcm-ntrip.org/svn/trunk/BNC.
        
        Abaixo estão alguns dos propósitos do BNC \citep{bnc}:
        \begin{itemize}
            \item recuperar fluxos de dados GNSS em tempo real disponíveis através do protocolo de transporte NTRIP;
            \item recuperar fluxos de dados GNSS em tempo real via TCP diretamente de um endereço IP sem usar o NTRIP protocolo de transporte;
            \item recuperar fluxos de dados GNSS em tempo real a partir de uma porta UDP ou serial local sem usar o NTRIP protocolo de transporte;
            \item gerar arquivos de Observação e Navegação RINEX de alta taxa para suportar aplicações GNSS pós-processadas quase em tempo real;
        \end{itemize}

\item Computador: Para a utilização dos softwares de processamento em tempo real e pós processamento o o computador ''Intel(R) Core (TM) i5-7200U CPU @ 2.50GHz'' será utilizado;

\item Receptor GNSS: para as medições em campo será utilizado o receptor ”Topcon GR5”, da Seção de Engenharia Cartográfica do Instituto Militar de Engenharia. Cabe ressaltar que esse possui comunicação via celular SIM Card, GSM/CDMA ou HSPA interno no receptor e suporte à tecnologia NTRIP, requisito para a realização do RTK/NTRIP.

\item Coletor de dados: Tablet Topcon FC-5000. Esse equipamento possibilita conexão direta com o GR-5, com um sistema operacional Windows 10, é um computador de bordo no campo de operações.
         
\end{itemize}


\section{Métodos}
\noindent

Os métodos escolhidos para esse projeto visam demonstrar a viabilidade da aplicação das técnicas de posicionamento utilizando GNSS em tempo real. Ademais, busca-se mostrar as etapas e ferramentas necessárias para cada método de posicionamento, visto que em projetos e aplicações de engenharia, a implementação inicial de métodos novos e pouco consolidados é uma etapa adicional em relação a métodos já estabelecidos.

Neste projeto serão obtidas as acurácias esperadas do RTK (RTPPP e RTK/NTRIP) e do PPP-RTK, bem como as etapas práticas necessárias para a sua realização, no sentido de fornecer as informações necessárias para a escolha de cada método adequado para cada situação específica.

\subsection{RTPPP com estação da RBMC}

\begin{itemize}
    \item Primeiramente, por intermédio do BNC, a avaliação do RTPPP será feita utilizando uma estação estática da Rede Brasileira de Monitoramento Contínuo, RBMC:
    \begin{itemize}
        \item Estação escolhida: Porto Alegre (POAL) – Estação pertencente à Rede de Densificação do IGS e à Rede de Referência do SIRGAS;
        \item Coordenadas em SIRGAS 2000 na época 2019.48:
        \begin{itemize}
            \item X: 3467519.4343 m;  Y: -4300378.65881 m; Z: -3177517.52227 m \citep{sirgascon}.
        \end{itemize}
        \item A escolha de uma estação estática para a primeira coleta e processamento de dados é fundamental para o entendimento prático do método:
            \begin{itemize}
            \item Coordenadas de alta precisão e confiabilidade para trabalhos futuros;
            \item Posição da estação simulando a ideal, elevada e com o mínimo de obstruções a fim de coletar o máximo de sinais;
            \item Antena \textit{choke ring} (TRM59800.00) a qual é a ideal para atenuar os efeitos de multicaminho.
            \end{itemize}

        \item O software coletará em tempo real, com um intervalo de 01 segundo durante o período de 24h os dados brutos (em formato RINEX) da estação POAL através da conexão:
        \begin{itemize}
            \item www.igs-ip.net, porta 2102, \textit{mountpoint} POAL00BRA0.
        \end{itemize}
        
        \item Simultaneamente com dados da estação POAL, o software coletará os dados das órbitas transmitidas e relógios dos satélites, bem como a solução dos relógios dos satélites para atenuar os erros de relógio. Tais dados serão coletados a partir das seguintes conexões:
        \begin{itemize}
            \item Caster 1: products.igs-ip.net, porta 2101, \textit{mountpoint} CLK91.
            \item Caster 2: products.igs-ip.net, porta 2101, \textit{mountpoint} RTCM3EPH.
        \end{itemize}
        \item Após a coleta, ter-se-ão as informações das coordenadas cartesianas da estação POAL para cada segundo, bem como as variações das coordenadas no sistema geodésico local (dE, dN, dU). Desta forma, os dados de dE, dN e dU são extraidos e constroi-se um gráfico de série temporal com esses dados.
    
    \end{itemize}
    \item Configuração do BNC para RTPPP - serão observados os seguintes detalhes:
    \begin{itemize}
        \item Indicação dos caminhos para armazenamento do \textit{log} do RTPPP e dos RINEX. Na aba ''PPP(1)'' seleção de ''\textit{Real-time streams}'' na opção de ''\textit{data-source}''. ''\textit{CLK91}'' em ''\textit{corrections stream}''. Indicação do caminho para o arquivo (igs14.atx) que contém as especificações de antenas disponível em \cite{ngsatx}.
        \item No arquivo com as coordenadas da estação utilizada deve-se especificar o modelo de antena do receptor.
    \end{itemize}
\end{itemize}

\subsection{RTK com GR-5}


Com o intuito de realizar as configurações iniciais do receptor, bem como verificar o correto armazenamento das observáveis obtidas pelo receptor. serão realizados um Posicionamento por Ponto e um RTK utilizando o marco do IBGE ''91780'', localizado no terraço do quinto andar do prédio frontal do IME.

\begin{itemize}
    \item Ambientação com o equipamento: Como o GR-5 é recém-chegado na Seção de Engenharia Cartográfica, é necessário realizar as configurações do aparelho, por exemplo: 
    \begin{itemize}
        \item Atualização de Firmware;
        \item Configuração de hemisfério (procedência estadunidense, ou seja, alterar a opção de hemisfério norte para sul, afim de rastrear os satélites corretamente);
        \item Fuso horário;
        \item Sistema de Referência;
        \item Comunicação via link de rádio;
        \item Configuração da controladora.
    \end{itemize}
    \item Testes Iniciais: Com o apoio da equipe de topografia da sessão, foram realizadas verificações do funcionamento básico do aparelho: sistema de coordenadas, bateria, conexão bluetooth à controladora e via porta serial com notebook;
    
    \item Levantamentos:
    
    Posicionamento por Ponto Estático e RTK: A fim de verificar a gravação de dados pela Base e a confirmação da coerente medição de pontos, optou-se por realizar um PP Estático. Arquivos RINEX de observação foram gerados para análise;

    
\end{itemize}


\section{Comparação entre os métodos}


Ao coletar dados por meio dos receptores e dos Casters e registrá-los em formato RINEX, pode-se realizar a avaliação da acurácia. Pós-processando esses dados, temos uma solução confiável para realizar as comparações. 


\section{Especificações técnicas}

Com o intuito de facilitar e disseminar a utilização dos métodos aqui estudados o presente trabalho visa fornecer especificações técnicas que tratem a respeito da realização dos métodos de RTPPP, PPP-RTK e RTK/NTRIP. Tal especificação deverá conter o seguinte:
\begin{itemize}
    \item especificação dos equipamentos utilizados em cada um dos métodos;
    \item esquema das medições;
    \item especificação dos softwares utilizados;
    \item etapas necessárias para a realização de cada método;
    \item etapas necessárias para o processamento em tempo real e pós-processamento dos dados; e
    \item acurácia esperada para cada método.
\end{itemize}

\section{Cronograma}
\noindent

Segue na Figura \ref{fig:cronograma} o cronograma com as atividades previstas para o desenvolvimento do Projeto. Considera-se como período útil à realização do projeto de 12 de fevereiro de 2019 a 17 de outubro de 2019.

\begin{figure}[H]
	\centering  
	\caption{Cronograma de Projeto}
	\label{fig:cronograma}
    
	\begin{ganttchart}[
    y unit title=0.4cm,
    y unit chart=0.5cm,
    vgrid,
    time slot format=isodate-yearmonth,
    compress calendar,
    title/.append style={draw=none, fill=verylightgray},
    title label font=\sffamily\bfseries\color{black},
    title label node/.append style={below=-1.6ex},
    title left shift=.05,
    title right shift=-.05,
    title height=1,
    bar/.append style={draw=none, fill=gray},
    bar height=.6,
    bar label font=\normalsize\color{black!50},
    group right shift=0,
    group top shift=.6,
    group height=.3,
    group peaks height=.2,
    bar incomplete/.append style={fill=silver},
    progress label text= %completo%    }%
   ]{2019-01}{2019-12}
   
    \gantttitlecalendar{year}\\
    \ganttgroup{Definição de Projeto}{2019-02}{2019-05} \\
        \ganttbar[progress=100, name=T1A]{Validação do Tema}{2019-02}{2019-05} \\
        \ganttbar[progress=100]{Definição da Metodologia}{2019-03}{2019-05} \\
    
    \ganttgroup{Revisão Bibliográfica}{2019-02}{2019-07} \\
        \ganttbar[progress=100, name=T2A]{Revisão sobre os métodos}{2019-02}{2019-07} \\
    
    \ganttgroup{Medições}{2019-07}{2019-09} \\
        \ganttbar[progress=100, name=T2A]{Medições em campo}{2019-07}{2019-09} \\
        
    \ganttgroup{Processamento dos dados}{2019-07}{2019-09} \\
        \ganttbar[progress=100]{Processamento}{2019-07}{2019-09}\\
        
    \ganttset{link/.style={green}}
    \ganttgroup{Análises de Resultados}{2019-08}{2019-10} \\
        \ganttbar[progress=100]{Comparações}{2019-08}{2019-10}\\
        
    \ganttgroup{Produção Textual}{2019-03}{2019-11} \\
        \ganttbar[progress=100]{Relatório I}{2019-03}{2019-05}\\
        \ganttbar[progress=100]{Relatório II}{2019-06}{2019-07}\\
        \ganttbar[progress=100]{Relatório III}{2019-08}{2019-10}\\
        %\ganttbar[progress=0]{Artigo*}{2019-10}{2019-10.5}\\
\end{ganttchart}
\end{figure}






São descritos a seguir de forma breve as atividades pretendidas para cada um dos itens descritos no cronograma apresentado:

\begin{enumerate}
    \item Definição de Projeto 
        \begin{enumerate}
            \item Validação do Tema: a etapa visa a validação da proposta do projeto, com a definição dos objetivos, verificação de trabalhos que tratem a respeito do assunto e a verificação e explanação da necessidade de realização do projeto.
            
            \item Definição da metodologia: definição da metodologia a ser seguida para que se alcancem os objetivos propostos.
        \end{enumerate}

    \item Revisão Bibliográfica
        \begin{enumerate}
            \item Revisão sobre os métodos: a etapa visa realizar a revisão bibliográfica sobre os métodos utilizados no presente trabalho assim como outros métodos relacionados que servirão de base para a confecção do relatório.
        \end{enumerate}

    \item Medições
        \begin{enumerate}
            \item Medições em campo: realização das medições com aparelho GNSS (modelo ''GR-5'') na Praça General Tibúrcio (Rio de Janeiro - RJ) e também com computador conectado a internet.
        \end{enumerate}
    
    \item Processamento
        \begin{enumerate}
            \item Processamento dos dados: realização do processamento em laboratório dos dados coletados na etapa de Medições a fim de gerar uma solução mais confiável que servirá de base para comparação e definição das acurácias alcançadas com os métodos. O processamento visa utilizar ferramentas de software livre e softwares licenciados de posse da Seção de Engenharia Cartográfica do Instituto Militar de Engenharia.
        \end{enumerate}

    \item Análise de Resultados
        \begin{enumerate} 
            \item Comparações: esta etapa visa comparar o resultados das medições em campo com o resultado das medições pós-processadas de forma que seja possível estabelecer uma precisão para as medições em campo.
        \end{enumerate}

    \item Produção Textual
        \begin{enumerate} 
            \item Relatórios: produção de relatórios parcial (I e II) e final (III).
            \item Artigo*: produção de artigo (a definir) acerca do tema desenvolvido ao longo do projeto.
        \end{enumerate}
\end{enumerate}