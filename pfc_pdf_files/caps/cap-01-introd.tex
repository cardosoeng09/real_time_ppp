\chapter{Introdução}

Os Sistemas de Navegação Global por Satélite, do inglês \textit{Global Navigation Satellite Systems (GNSSs)}, possibilitam a estimativa da posição geográfica de um ponto em qualquer lugar da superfície terrestre e em suas proximidades. Suas aplicabilidades vão muito além do mapeamento, como sincronização de relógios ao redor do globo, navegação, segurança, agricultura de precisão, aviação civil, meio ambiente, forças armadas, entre outras. \citep{liu2018real} \citep{monico2008}.

Os sistemas de navegação por satélites atualmente disponíveis, como GPS, GLONASS, Galileu, Beidou-Compass estão em constante processo de evolução e modernização, englobando os seguimentos de controle e espacial com a transmissão de sinais modernizados no qual se espera melhorias na acurácia posicional e na integridade dos sistemas.Os métodos de posicionamento por GNSS são aprimorados com o passar dos anos em função da evolução tecnológica e da demanda por maior acurácia e versatilidade.

Os métodos de posicionamento por GNSS, basicamente, podem ser classificados em \textit{posicionamento absoluto}, quando as coordenadas estão associadas diretamente ao geocentro e \textit{posicionamento relativo}, quando as coordenadas são determinadas com relação a um referencial materializado por um ou mais vértices com coordenadas conhecidas \citep{monico2008}.

Neste Projeto Final de Curso, três métodos serão aplicados os métodos: o RTPPP (\textit{Real Time Precise Point}), RTK (\textit{Real Time Kinematic}) e o PPP-RTK (integração RTPPP e RTK).


O RTPPP envolve a aplicação de PPP (Posicionamento Por Ponto Preciso) em Tempo real) e requer somente um receptor ao nível do usuário, diferente do RTK, que requer observações e correções de uma base.  Ao utilizar-se de conexão com internet, com um software de processamento em tempo real, é possível receber as órbitas transmitidas dos satélites bem como a correção das órbitas e relógios. \citet{marques2014ppp} ressaltam que o método RTPPP encontra-se em desenvolvimento para aplicação em larga escala até nos dias atuais e cita melhorias que podem ser buscadas, tais como a diminuição no tempo de convergência e a utilização da solução fixa de ambiguidades ao invés da solução float, o que é factível no posicionamento relativo. A acurácia esperada do método RTPPP é na ordem decimétrica \citep{lima}.

No caso do RTK, diversos trabalhos têm mostado a potencialidade do método para aplicações em tempo real, onde se pode destacar o RTK em rede com aplicações de estações virtuais \citep{daniele}. O padrão de transmissão no RTK, em geral, utiliza o formato RTCM com envio a partir de link de rádio, contudo, com o desenvolvimento do protocolo NTRIP, diversos receptores na atualidade permitem a inserção de chips com tecnologia GSM e acesso à internet de tal forma que se pode utilizar o protocolo NTRIP e a recepção de dados de estações de uma rede de monitoramento contínuo. A utilização de tal protocolo no RTK tem levado a comunidade usuária no Brasil à classificar o método como RTK/NTRIP.


Nesse trabalho de PFC foram aplicados os métodos RTPPP, PPP-RTK e RTK com envio de dados a partir de link de radio e também via NTRIP. Para aplicação dos métodos foi utilizado o receptor GR-5 da Topcon, sendo este equipamento recém chegado na seção de ensino 6 do IME, o que demandou considerável tempo do alunos para o entendimento e treinamento com o método. Para aplicação do RTPPP foi utilizado o software BNC disponibilizado pelo centro Alemão BKG e a solução foi realizada para estação estática da RBMC, bem como com conexão do software no receptor GR-5 via porta serial. O método PPP-RTK foi realizado utilizando o software PPP-Wizard disponibilizado pelo CNES (Centre National d'Études Spatiales) da França, sendo que este permite receber diversas informações da rede mundial do IGS e possibilita a solução fixa das ambiguidades. 
Os resultados obtidos nos levantamentos em campo foram analisados a partir da série temporal de estimativa de precisões, bem como dos valores de Erro Médio Quadrático (EMQ), ao comparar a solução em tempo real com solução pós-processada ou com coordenadas oficiais das estações. 




\section{Motivação}
\label{motivacao}
\noindent

A motivação do trabalho surge a partir de algumas questões e demandas. Como realizar um posicionamento por GNSS, em tempo real, utilizando os métodos de posicionamento RTPPP, RTK e PPP-RTK? Quais são as ferramentas livres na atualidade e como utilizar para a aplicações desses métodos? A acurácia posicional proporcionada pelos métodos de posicionamento em tempo real atende a demanda por levantamentos precisos e de navegação? A busca pelas respostas instiga o projeto

O crescimento da demanda por serviços de posicionamento GNSS no Brasil e no mundo aumenta a necessidade de documentação técnica a respeito do tema. A possibilidade de ter normas e padronizações a respeito do tema é interessante pois possibilita o uso de forma facilitada dos referidos métodos em diversas aplicações, tais como:

\begin{itemize}
    \item Monitoramento de frotas em tempo real;
    \item Monitoramento da posição de agentes de segurança pública em operações;
    \item Utilização pelas Forças Armadas para monitoramento em operações;
    \item Georreferenciamento de imóveis rurais;
    \item Levantamento de coordenadas para construção civil;
    \item Obtenção de coordenadas precisas de maneira eficiente para construção de rodovias;
    \item Utilização em qualquer aplicação que requeira métodos de posicionamento em tempo real com precisão centimétrica, etc.
\end{itemize}



Em 2013 foi publicada no Brasil a ''Norma Técnica para georreferenciamento de Imóveis Rurais'' \citep{ibge_imoveis}. Essa norma especifica os métodos de posicionamento que podem ser utilizados para o georreferenciamento de imóveis rurais no Brasil. Entre os métodos previstos estão o PPP em tempo real e RTK em rede (RTK/NTRIP), métodos que este trabalho se propõe a avaliar.

De acordo com o decreto nº 4.449, de 30 de outubro de 2002, e o decreto 9.311, de 15 de março de 2018, os proprietários de imóveis rurais de 25 a 100 hectares possuem até 20/11/2023 para apresentarem, junto ao governo federal, os georreferenciamentos de suas respectivas propriedades; os proprietários de imóveis com área inferior a 25 hectares possuem até 20/11/2025.

Embora os referidos decretos estejam em vigor e o ''Manual Técnico de Posicionamento: georreferenciamento de imóveis rurais'' publicado, é possível verificar a pouca utilização dos métodos de posicionamento por GNSS pelos grupos e empresas. Muitas das vezes pela resistência às novas tecnologias, falta de equipamento adequado e conhecimento. Dessa forma, busca-se disseminar as técnicas estudadas e apresentar de maneira confiável as acurácias que podem ser obtidas com cada uma.

Além da área de posicionamento para georreferenciamento de imóveis rurais existe também o interesse nesse tipo de posicionamento em empresas diversas de transporte e no setor de segurança pública. Uma empresa com frota de transporte rodoviário pode, utilizando os métodos aqui estudados, monitorar suas viaturas em tempo real com precisão decimétrica ou centimétrica de maneira confiável.

Indo além, órgãos de segurança pública e Forças Armadas podem rastrear, em tempo real, com a precisão citada, diversos escalões, tais como:
\begin{itemize}
    \item Pelotão em atividade de Garantia da Lei e da Ordem;
    \item Viaturas de transporte de suprimento ou pessoal;
    \item Barcos e embarcações diversas para transporte de suprimento ou pessoal;
    \item Grupos de agentes de segurança pública em atividades diversas;
    \item Viaturas das Polícias Militares;
    \item Monitoramento da frota de transporte público de determinada cidade; etc
\end{itemize}

Por fim, pela carência de literatura na área no Brasil, tais métodos carecem de normas técnicas e manuais que possam apresentar os passos necessários para realizar na prática tais métodos de posicionamento. Suprir essa carência é essencial para a disseminação do conhecimento. 


\section{Objetivo}
\noindent
O presente Projeto Final de Curso tem como objetivo geral:

Aplicar e avaliar a acurácia de levantamento a partir dos métodos de posicionamento em tempo real RTPPP, RTK e PPP-RTK fazendo uso de receptores GNSS modernos, dados de redes geodésicas de monitoramento contínuo e plataformas computacionais livres.


Objetivos Específicos:
\begin{itemize}
    \item Treinamento com o par de receptores GR-5 da Topcon adquirido recentemente pelo IME;
    \item Levantamento RTK padrão utilizando a comunicação via link de rádio;
    \item RTPPP com o BNC/BKG em estação estática da RBMC;
    \item RTPPP com o BNC/BKG e conexão com o receptor GR5 a partir da porta serial;
    \item PPP-RTK fazendo uso do aplicativo PPP-Wizard fornecido pelo CNS;
    \item Avaliação de acurácia dos levantamentos realizados.
\end{itemize}


