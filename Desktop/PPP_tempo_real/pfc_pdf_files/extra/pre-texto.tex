%%
%
% ARQUIVO: pre-texto.tex
%
% VERSÃO: 1.0
% DATA: Maio de 2016
% AUTOR: Coordenação de Trabalhos Especiais SE/8
% 
%  Arquivo tex para a criação da parte pré-textual do documento de Projeto de Fim de Curso.
%
%%


% -----
% PÁGINA DE CAPA DO DOCUMENTO DE PFC
% -----
\makecapa

% -----
% PÁGINA DE TÍTULO DO PFC
% -----
\prepareadvisors
\maketitle

% -----
% PÁGINA DE CRÉDITOS DO DOCUMENTO DE PFC
% -----
\makecredits

% -----
% PÁGINA DE FOLHA DE ASSINATURAS
% -----
\preparemembers
\approvalpage

% -----
% PÁGINA DE DEDICATÓRIA (OPCIONAL, ie. pode remover toda a página)
% -----
%%% DEDICATÓRIA - PREENCHER...
\dedicatoria{%
À Deus que sempre guia nossos passos pelo melhor caminho; Às nossas famílias que sempre estiverem presentes para apoiar-nos e Ao Instituto Militar de Engenharia, alicerce de nossa formação e aperfeiçoamento profissional.}%
\makededication

% -----
% PÁGINA DE AGRADECIMENTOS (OPCIONAL, ie. pode remover toda a página)
% -----
%%% AGRADECIMENTOS - PREENCHER...
\agradecimentos{%
É chegado ao fim de um ciclo, o qual se iniciou há muitos anos atrás, por meio de constantes desafios e conquistas. Sendo assim, agradecemos aos apoiadores dessa trajetória, nossa família, alicerce fundamental de vida e todos os professores envolvidos na caminhada, especialmente os nossos orientadores Profª Dra. Heloísa Alves Silva Marques e Profº Dr. Haroldo Antônio Marques. Gostaríamos de agradecer também ao 2º Ten Marco e ao Cap R1 Eduardo, que nos prestaram grande apoio durante as atividades práticas desse projeto estando sempre disponíveis nos momentos que foram necessários.}%
\makethanks

% -----
% PÁGINA DE EPÍGRAFE (OPCIONAL, ie. pode remover toda a página)
% -----
%%% EPÍGRAFE - PREENCHER...
\epigrafe{%
Não existe triunfo sem perda, não há vitória sem sofrimento e não há liberdade sem sacrifício.
}%
\autorepigrafe{%    %% Se não tem autor, coloque "Anônimo"
J. R. R. Tolkien
}%
\makeepigraph

% -----
% PÁGINA DE SUMÁRIO
% -----
\tableofcontents

% -----
% PÁGINAS DE LISTAS DE FIGURAS E DE TABELAS
% se a Dissertação não possui figuras e/ou tabelas, REMOVA O COMANDO CORRESPONDENTE
% -----
\listoffigures
%\listoftables

% -----
% PÁGINA DE LISTA DE SIGLAS
% se a Dissertação não possui siglas, REMOVA TODA A PÁGINA
% -----
%%% SIGLAS - PREENCHER...

\acronimo{GPS}{Global Positioning System}
\acronimo{DGPS}{Differential Global Positioning System}
\acronimo{GNSS}{Global Navigation Satellite System}
\acronimo{PPP}{Precise Point Positioning}
\acronimo{RTPPP}{Real-Time Precise Point Positioning}
\acronimo{PPP-RTK}{Precise Point Positioning Real-Time Kinematic}
\acronimo{IAG}{International Association of Geodesy}
\acronimo{IGS}{International GNSS Service}
\acronimo{RTCM}{Radio Technical Commission for Maritime Services}

\listofnicks


% -----
% PÁGINA DE LISTA DE ABREVIATURAS
% se a Dissertação não possui abreviaturas ou símbolos, REMOVA TODA A PÁGINA
% -----
%%% ABREVIATURAS
%\abreviatura{iid}{Independente e Identicamente Distribuído}
%\abreviatura{MA}{\textit{Moving Average}}


%%% SÍMBOLOS - PREENCHER...
%\simbolo{$\mu$}{média}
%\simbolo{$\gamma$}{covariância}
%\simbolo{$\rho$}{correlação}


%\listofsymbols

% -----
% PÁGINA DE RESUMO
% -----
%%% RESUMO - PREENCHER...
\resumo{%
O presente relatório visa apresentar as etapas do projeto envolvendo posicionamento preciso em tempo real por GNSS utilizando ferramentas livres e aplicação dos seguintes métodos: Método Relativo Cinemático (RTK),Posicionamento por Ponto Preciso em Tempo Real (RTPPP) e integração de RTK com PPP, denominado PPP-RTK. O método RTK é bem conhecido nas áreas de engenharia e, em geral, utiliza-se a transmissão de correções em tempo real via link de rádio, contudo, atualmente é possível utilizar a transmissão de dados via RTCM utilizando protocolo NTRIP. O método RTPPP é relativamente novo e requer correções de órbitas (posição dos satélites) e relógios em tempo real, o que pode ser obtido via protocolo NTRIP utilizando, por exemplo, os serviços gratuitos do IGS. Ao receber as correções em tempo real de uma rede de estações GNSS, o método RTPPP permite fixar as ambiguidades da fase da portadora e passa a ser denominado PPP-RTK. Esses métodos (RTPPP e PPP-RTK) atualmente não são totalmente operacionais, haja vista que a maioria dos receptores GNSS não estão preparados para a recepção de dados em tempo real, além de outras questões, como por exemplo, a disponibilidade de internet em diversas regiões do Brasil. Uma possibilidade é utilizar serviços pagos, tal como o OminiSTAR-XP e Trimble-RTX, os quais enviam correções em tempo real a partir de satélites geoestacionários. Dessa forma, o objetivo desse projeto final de curso é aplicar os métodos RTK, RTPPP e PPP-RTK utilizando plataformas livres e verificar a qualidade posicional visando o monitoramento em tempo real, o qual será útil para monitorar frotas, estimar posições de embarcações e aeronaves, monitorar estruturas, entre outras aplicações. 
}%
\makeresumo

\abstract{%
This project aims to present the design steps involving GNSS real-time accurate positioning using free tools and applying the following methods: Relative Kinematic Method (RTK), Real-Time Precise Point Positioning (RTPPP) and integration of RTK with PPP, called PPP-RTK. The RTK method is well known in the engineering field and radio link real-time correction is generally used, however it is currently possible to use data transmission via RTCM using NTRIP protocol. The RTPPP method is relatively new and requires real time orbit (satellite position) and clock correction, which can be obtained via the NTRIP protocol using, for example, free IGS services. When receiving real-time corrections from a network of GNSS stations, the RTPPP method allows the carrier phase ambiguities to be fixed and is renamed PPP-RTK. These methods (RTPPP and PPP-RTK) are currently not fully operational, as most GNSS receivers are not prepared for real-time data reception, as well as other issues such as internet availability in various regions of Brazil. One possibility is to use paid services, such as OminiSTAR-XP and Trimble-RTX, which send real-time corrections from geostationary satellites. Thus, the purpose of this final course project is to apply RTK, RTPPP and PPP-RTK methods using free platforms and verify positional quality for real-time monitoring, which will be useful for monitoring fleets, estimating vessel and aircraft positions. , monitor structures, among other applications.
}%
\makeabstract
